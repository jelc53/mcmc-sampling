\documentclass[12pt,letterpaper,twoside]{article}

\newif\ifsolution\solutiontrue   % Include the solutions
%\newif\ifsolution\solutionfalse  % Exclude the solutions

\usepackage{stats370}
\usepackage{xcolor}
\usepackage{float}
\usepackage{dsfont}
\usepackage{enumitem}
\usepackage{mathtools}
% \usepackage{breqn}
\usepackage{graphicx}
\usepackage[section]{placeins}
% \graphicspath{{output/}{\main/output/}}

\allowdisplaybreaks
\raggedbottom

\newcommand{\T}[1]{\text{\texttt{#1}}}
\newcommand{\V}[1]{\text{\textit{#1}}}

\begin{document}

{\centering \textbf{Project: Sampling Methodologies \\}}
\vspace*{-8pt}\noindent\rule{\linewidth}{1pt}

\paragraph{Purpose} of the project is to compare different sampling methodologies 
for estimating the posterior $p(\theta|y_1,...,y_n)$ of a bayesian 
inference problem.

\paragraph{Data} $(y_1,...,y_n)$ are gene expression measurements for two genes 
on $n=24$ samples where $y_i=(y_{i,1}, y_{i,2})$ represent gene expressions for 
sample i. Our samples are all labelled by "group", denoted $(t_1,...,t_n)$. 
Cell type (or mix of cell types) vary with group and we assume mean gene 
expression (but not variance) depends on cell type. Moreover, the expressions 
of different genes are independently normally distributed.

\begin{itemize}
    \item $Y_i \sim N(\mu, \sigma^2 \mathbb{I})$ if sample $i$ from group 1
    \item $Y_i \sim N(\gamma, \sigma^2 \mathbb{I})$ if sample $i$ from group 2
    \item $Y_i \sim N(0.5\mu + 0.5\gamma, \sigma^2 \mathbb{I})$ if sample $i$ from group 3
    \item $Y_i \sim N(\tau\mu + (1-\tau)\gamma, \sigma^2 \mathbb{I})$ if sample $i$ from group 4
\end{itemize}

Our model has a 6-dimensional parameter $\theta = (\sigma^2, \tau, \mu_1, \mu_2, \gamma_1, \gamma_2)$.

For all sampling methodologies, I computed 5000 samples with a 
burn-in of 200. The rest (step size, momentum, proposals, etc.) I 
discuss and treat as design choices.

\section{Metropolis Hastings}
To implement the Metropolis-Hastings algorithm we require the 
joint and transition densities to compute the hastings ratio. 
Below we derive the join density for our model, and treat the 
transition density as a design choice.

\textbf{Joint density}:
\begin{align*}
    p(\theta|y) & \propto \prod_{i=1}^n p(y|\theta) p(\theta) \\
                & = \prod_{t=1}^4 \prod_{i=1}^n 1\{t_i=t\} \cdot p(y_i|\theta) \cdot p(\mu) \cdot p(\gamma) \cdot p(\tau) \cdot p(\sigma^2) \\
                & = \prod_{t_i=1} N(\mu, \sigma^2 \mathbb{I}) \cdot \prod_{t_i=2} N(\gamma, \sigma^2 \mathbb{I}) \cdot \prod_{t_i=3} N(0.5\mu + 0.5\gamma, \sigma^2 \mathbb{I}) \cdot \prod_{t_i=4} N(\tau\mu + (1-\tau)\gamma, \sigma^2 \mathbb{I}) \cdot p(\sigma^2)
\end{align*}

\textbf{Hastings Ratio}: let $q(\theta_0, \theta_1)$ be our transition density and 
$\pi(y,\theta)$ be our join density. Then, 
$$\text{accept prob.} = \min(1, \frac{q(\theta_0, \theta_1)\pi(y,\theta_1)}{q(\theta_1, \theta_0)\pi(y,\theta_0)})$$

\begin{python}
def metropolis_hastings(data, n_samples, step_size, inital_position):
    curr_theta = inital_position.copy()
    samples = [curr_theta]

    while it < n_samples:

        proposed = proposal_sampler(curr_theta, step_size)
        h_ratio = hastings_ratio(proposed, curr_theta, data)
        accept_prob = min(1, h_ratio)

        if np.random.uniform(0, 1) <= accept_prob:
            curr_theta = proposed
            accept_count += 1

        samples.append(curr_theta)

    return samples
\end{python}

The Metropolis-Hastings algorithm itself is very simple,
However, there are still a number of design choices we need to make, 
such as (a) how to propose a new sample, (b) enforcing any constraints 
we have (e.g. $\tau \in [0,1]$), and (c) choosing hyperparameters (step 
size, burn-in).

\begin{enumerate}[label=(\alph*)]
\item \textbf{Proposal selection}: Tried two methods: symmetric with constraints
imposed as part of acceptance criteria, and asymmetric with constraints encoded 
into proposal distributions.

The symmetric method was attractive from an implementation perspective, since 
it would guarantee detail balance (symmetric transition kernel) and therefore 
simplify our hastings ratio to $\frac{\pi(y, \theta_1)}{\pi(y,\theta_0)}$ since 
$q(\theta_0, \theta_1)=q(\theta_1, \theta_0)$. I picked a gaussian random step 
with mean equal to step size and variance of 1.

\begin{python}
def proposal_sampler(curr_theta, step_size):
    num_params = len(curr_theta)

    while True:
        randomness = np.random.normal(0, 1, num_params)
        proposal = curr_theta + np.multiply(step_size, randomness)

        if proposal[0] > 0 and (proposal[1] >= 0 and proposal[1] <= 1):
            break

    return proposal
\end{python}

The asymmetric method made use of truncated normals that directly encoded 
the bounds of our problem in the proposal distributions. This required that 
I also solve for an asymmetric transition kernel and evaluate the full 
hastings ratio, but the resulting algorithm was much more efficient (see 
constraints section below).
\begin{python}
def proposal_sampler(curr_theta, step_size):
    proposed = np.zeros_like(curr_theta)
    a, b = (0 - proposed[0])/step_size, (np.inf - proposed[0])/step_size
    c, d = (0 - proposed[1])/step_size, (1 - proposed[1])/step_size

    proposed[0] = truncnorm(a, b, proposed[0], step_size).rvs()
    proposed[1] = truncnorm(c, d, proposed[1], step_size).rvs()
    proposed[2] = norm(proposed[2], step_size).rvs()
    proposed[3] = norm(proposed[3], step_size).rvs()
    proposed[4] = norm(proposed[4], step_size).rvs()
    proposed[5] = norm(proposed[5], step_size).rvs()

    return proposal
\end{python}

\item \textbf{Enforcing constraints}: We had two constraints to consider: (a) $\sigma^2 > 0$ 
and (b) $0 \le \tau \le 1$. I enforce these constraints for our symmetric proposal method by resampling 
if the proposal violates either (a) or (b). This reduces the efficiency (acceptance ratio) of 
the algorithm substantially and introduces potential bias in our resulting marginal posterior 
samples (particularly for $\tau$ which has significant density near its upper bound, mean shifted 
from ~0.75 to ~0.85). By contrast, our asymmetric proposal method encodes these constraints directly. 
For these reasons, we use the asymmetric kernel for our results.

\item \textbf{Hyperparameters}: There are not many hyperparameters to 
tune for Metropolis-Hastings, which is one of its advantages! I 
experimented with different step sizes (0.01, 0.05, 0.1, 0.5), 
and for each computed the acceptance ratio and number of 
effective (uncorrelated) samples for 300 sample test (after 
burn-in). I choose a step size of 0.05 since it maximized 
both number of effective samples and acceptance ratio.

\begin{table}[H]
    \centering
    \begin{tabular}{lllll}
        Step size                   & 0.01 & 0.05  & 0.1   & 0.5   \\
        Acceptance ratio            & 0.87 & 0.44  & 0.22  & 0.02  \\
        Mean effective samples      & 3.5  & 16.6  & 14.6  & 6.96  \\
                                    &      &       &       &         
    \end{tabular}
\end{table}

I did not investigate varying initialization (used $[1., 0.5, 0., 0., 0., 0.]$)
or burn-in (fixed at 200 to match what I saw in textbook examples).
\end{enumerate}

\textbf{Results}: Using the above design choices, we produce the following 
samples from our posterior (figure 1) and compute the first and second 
moments (table below). These passed basic sense checks, such as $\sigma^2$ 
and $\tau$ falling within their respective bounds, mean and variance matching 
results from other sampling algorithms, and lower variance as we increase number 
of samples.
\begin{figure}[H]
    \centering
    \includegraphics[scale=0.55]{mh_sampled_histogram.png}
    \caption{Histogram of Metropolis-Hastings posterior samples}
\end{figure}

\begin{table}[H]
    \begin{tabular}{lcccccc}
    \multicolumn{1}{c}{}          & sigma & tau   & mu1   & mu2   & gam1  & gam2  \\
    Mean of posterior samples     & 0.14  & 0.84  & -1.44 & -0.67 & -0.25 & 0.35  \\
    Variance of posterior samples & 0.004 & 0.008 & 0.015 & 0.023 & 0.022 & 0.025
    \end{tabular}
\end{table}

The traceplots in figure 2 give an indication of how well the sampler 
explored the posterior density. I am comforted by the fact that 
there are no long periods of repeated values (getting stuck) 
despite an acceptance rate of only 0.44.
\begin{figure}[H]
    \centering
    \includegraphics[scale=0.55]{mh_sampled_traceplot.png}
    \caption{Traceplot for Metropolis-Hastings sampling}
\end{figure}

\textbf{Final remarks}. Metropolis-Hastings is simple to implement 
and relatively fast to run (time = 197 sec for 5000 samples), however,
number of effective samples (autocorrelation) and acceptance ratio 
were not as good as we might expect from other technqiues that take 
more directed paths through the posterior space (e.g. HMC).   



\section{Hamiltonian Monte Carlo} 
To implement the Hamiltonian Monte Carlo we need to compute 
the Hamiltonian $H(\rho, \theta)$ given independently drawn momentum 
$\rho$ and our current paramewter values $\theta$. 
\begin{align*}
    H(\theta, y) & = -\log p(\rho, \theta) \\
                 & = - \log p(\rho|\theta) - \log p(\theta) \\
                 & = K(\rho, \theta) + V(\theta) && \text{kinetic and potential energy}
\end{align*}
We now evolve the system using the following equations from 
Hamiltonian mechanics. Note: $\frac{\partial V}{\partial \rho} = 0$ 
and if we choose our kinetic energy to be gaussian 
$K(\rho, \theta) = \frac{1}{2}\rho^T M^{-1} \rho + \log |M| + \text{const.}$ 
so we get $\frac{\partial K}{\partial \rho} = M^{-1} \rho$ and 
$\frac{\partial K}{\partial \theta} = 0$.
\begin{align*}
    \frac{d\theta}{dt} & = + \frac{\partial H}{\partial \rho} = + \frac{\partial K}{\partial \rho} + \frac{\partial V}{\partial \rho} = M^{-1} \rho \\
    \frac{d\rho}{dt}   & = - \frac{\partial H}{\partial \theta} = -\frac{\partial K}{\partial \theta} - \frac{\partial V}{\partial \theta} = -\frac{\partial V}{\partial \theta}
\end{align*}

Therefore, to implement the above system, we need only to calculate 
the gradient of our potential energy (negative log likelihood) with 
respect to our parameters $\theta$.

\textbf{Negative log likelihood}: let $V(\theta) = -l(\theta) = -\log p(\theta|y)$,
\begin{align*}
    V(\theta) & = - \log \prod_{i=1}^n p(y|\theta) p(\theta) \\
              & = - \sum_{i=1}^n \log p(y|\theta) - \log p(\theta) && \text{let $y_i'$ = centered data}\\
              & = - \sum_{i=1}^n \log \left[(2\pi)^{-k/2} \cdot |\sigma^2 \mathbb{I}|^{-1/2} \cdot \exp(\frac{-1}{2} y_i'^T (\sigma^2 \mathbb{I})^{-1} y_i')\right] - \log \frac{1}{\sigma^2} \\
              & = \frac{nk}{2} \log 2\pi + (n+1) \log \sigma^2 + \frac{1}{2\sigma^2} \sum_{i=1}^n y_i'^T \mathbb{I}^{-1} y_i'       
\end{align*}

\textbf{Gradient of $V(\theta)$ wrt $\theta$}: 
\begin{align*}
    \frac{\partial V(\theta)}{\partial \sigma^2} & = \frac{n+1}{\sigma^2} - \frac{1}{2(\sigma^2)^2}\sum_{i=1}^n \|y_i'\|^2_2 \\
    \frac{\partial V(\theta)}{\partial \tau} & = \frac{1}{\sigma^2} \cdot (\gamma - \mu)^T \sum_{t_i=4} y_i' \\
    \frac{\partial V(\theta)}{\partial \mu} & = - \frac{1}{\sigma^2} \cdot \left[\sum_{t_i=1} y_i' + 0.5\sum_{t_i=3} y_i' + \tau\sum_{t_i=4} y_i'\right]\\
    \frac{\partial V(\theta)}{\partial \gamma} & = - \frac{1}{\sigma^2} \cdot \left[\sum_{t_i=2} y_i' + 0.5\sum_{t_i=3} y_i' + (1-\tau)\sum_{t_i=4} y_i'\right]
\end{align*}

Our implementation of these hamiltonian equations employs a 
leapfrog algorithm to perform numerical integration. This is 
a second-order method as opposed to euler which is first-order. 
This design choice helps with stability.
\begin{python}
def leapfrog(q, p, data, M_mat, path_len, step_size):
    q, p = np.copy(q), np.copy(p)  # curr_theta = q

    p -= step_size * dVdq(data, q) / 2
    for _ in range(int(path_len / step_size)):
        q, p = q_update(q, p, M_mat, step_size)
        p -= step_size * dVdq(data, q)

    q, p = q_update(q, p, M_mat, step_size)
    p -= step_size * dVdq(data, q) / 2

    # momentum flip at end
    return q, -p
\end{python}

\begin{python}
def hamiltonian_monte_carlo(n_samples, initial_position, m, step_size, path_len):
    samples, theta = [], inital_position
    size = (n_samples,) + initial_position.shape[:1]
    M_mat = np.eye(len(initial_position)) * m
    momentum = st.norm(0, m)

    for p0 in momentum.rvs(size=size):
        # integrate over our path to get new position and momentum
        q_new, p_new = leapfrog(
            theta,
            p0,
            data,
            M_mat=M_mat,
            path_len=path_len,
            step_size=step_size,
        )

        # check metropolis acceptance criterion
        curr_u = -log_likelihood(data, theta) #- beta(20, 3).logpdf(theta[1])
        prop_u = -log_likelihood(data, q_new) #- beta(20, 3).logpdf(q_new[1])
        curr_k = (0.5/m)*np.linalg.norm(p0,2)**2
        prop_k = (0.5/m)*np.linalg.norm(p_new,2)**2
        log_hratio = curr_u - prop_u + curr_k - prop_k
        accept_log_prob = min(0, log_hratio)

        if np.log(np.random.uniform(0, 1)) <= accept_log_prob:
            samples.append(q_new)
            accept_count += 1
            theta = q_new
        else:
            samples.append(np.copy(samples[-1]))

    return np.array(samples)
\end{python}

The Hamiltonian Monte Carlo algorithm has a number of design 
choices, including (a) whether to use an acceptance criterion, 
(b) enforcing any constraints we have (e.g. $\tau \in [0,1]$), 
and (c) choosing hyperparameters (step size, momentum).

\begin{enumerate}[label=(\alph*)]
\item \textbf{Acceptance criterion}: My first implementation did not 
include an acceptance step which was faster to evaluate but did 
not produce sensible, unbiased posterior samples. Adding the 
metropolis acceptance criterion at the end of each iteration 
helped to correct for errors introduced via numerical 
integration.

\item \textbf{Enforcing constraints}: We had two constraints to consider: 
(a) $\sigma^2 > 0$ and (b) $0 \le \tau \le 1$. I experimented with 
two methods of enforcing these constraints: boundary reflection (from 
particle physics applications) and imposing a beta prior on $\tau$.

For the boundary reflection method, I enforce these 
constraints in the q ($\theta$) update function by creating a 
"hard wall" or barrier, where we assume both $p$ (momentum) and 
$q$ (parameter $\theta$) are reflected (flip sign) by same magnitude 
that they surpassed the constraint. While this method did seem to work 
(resulting posterior marginals had similar first and second moments), it 
was less efficient (lower effective acceptance ratio).  

\begin{python}
def q_update(q, p, M_mat, step_size):
    """Helper function to update theta within bounds"""
    q_prop = q + step_size * np.linalg.inv(M_mat) @ p

    # check bounds
    if q_prop[0] < 0:
        p[0] = -p[0]
        q_prop[0] = -q_prop[0]

    if q_prop[1] < 0 or q_prop[1] > 1:
        p[1] = -p[1]
        q_prop[1] = -q_prop[1]

    return q_prop, p
\end{python}

Imposing a beta prior on $\tau$ ($\text{Beta}(20,3)$) not only encoded the 
constraints we wanted into our problem but also made the resulting joint 
posterior space differentiable, and therefore easier to traverse and sample 
from with gradient steps. This was the method I ended up using to produce my 
results. Note, this requied we both update our log likelihood and gradient 
calculations to reflect the new beta prior (rather than Unif[0,1]).
$$ V(\theta) = \frac{nk}{2} \log 2\pi + (n+1) \log \sigma^2 + \frac{1}{2\sigma^2} \sum_{i=1}^n y_i'^T \mathbb{I}^{-1} y_i' + \log({\text{B}(\alpha, \beta)}) - \log(\tau^{\alpha-1}(1-\tau)^{\beta-1}) $$
$$ \frac{\partial V(\theta)}{\partial \tau} = \frac{1}{\sigma^2} \cdot (\gamma - \mu)^T \sum_{t_i=4} y_i' + \frac{1-20}{\tau} + \frac{2}{1-\tau} $$ 

\item \textbf{Hyperparameters}: I experimented with different step 
sizes (0.005, 0.01) and momentum variance values (1, 5, 10). 
For each combination I computed the acceptance ratio and number of 
effective (uncorrelated) samples for 500 sample test (after 
burn-in of 200). I choose a (step size, momentum) pair of 
(0.01, 5) since it maximized number of effective samples while 
maintaining reasonable acceptance. Note, for higher step sizes 
(e.g 0.05) my algorithm would run into numerical instability issues 
so I did not test further in this direction.

\begin{table}[H]
    \centering
    \begin{tabular}{llll}
        Step size    & Momentum     & Accept ratio      & Mean eff. samples     \\
        0.005        & 1            & 0.97              & 138                   \\ 
        0.005        & 5            & 0.83              & 159                   \\
        0.005        & 10           & 0.80              & 30                    \\
        0.01         & 1            & 0.92              & 124                   \\
        0.01         & 5            & 0.76              & 195                   \\
        0.01         & 10           & 0.69              & 10                    \\    
    \end{tabular}
\end{table}

I did not investigate varying initialization (used $[1., 0.8, 0., 0., 0., 0.]$), 
burn-in (fixed at 200 to match what I saw in textbook examples) or path length
(used 1 as set number of leapfrog steps eual to path length / step size).
\end{enumerate}

\textbf{Results}: Using the above design choices, we produce the following 
samples from our posterior (figure 3).
\begin{figure}[H]
    \centering
    \includegraphics[scale=0.55]{hmc_sampled_histogram.png}
    \caption{Histogram of Hamiltonian Monte Carlo posterior samples}
\end{figure}

\begin{table}[H]
    \begin{tabular}{lcccccc}
    \multicolumn{1}{c}{}          & sigma & tau   & mu1   & mu2   & gam1  & gam2  \\
    Mean of posterior samples     & 0.13  & 0.88  & -1.42 & -0.65 & -0.28 & 0.33  \\
    Variance of posterior samples & 0.001 & 0.003 & 0.013 & 0.012 & 0.024 & 0.022
    \end{tabular}
\end{table}

The traceplots in figure 4 give an indication of how well the sampler 
explored the posterior density. I am comforted by the fact that 
there are no long periods of repeated values (getting stuck) 
despite us not implementing no-turn logic. 
\begin{figure}[H]
    \centering
    \includegraphics[scale=0.55]{hmc_sampled_traceplot.png}
    \caption{Traceplot for Hamiltonian Monte Carlo sampling}
\end{figure}

\textbf{Final remarks}. HMC was much slower to run than Metropolis-
Hastings due to the gradient evaluation at each iteration (4,205 sec 
for 5000 samples), however, our acceptance ratio was much higher 
(approx. ~0.85 vs ~0.45). Moreover, our samples were far less likely 
to be autocorrelated (approx. 32 vs 5 effective samples per 100 
iterations). This is because we are taking directed rather than 
random steps through the posterior.



\section{Gibbs Sampling}
To implement the Gibbs Sampling we need to derive the conditional 
distributions for each parameter from which to sample. 

\textbf{Conditional distributions} Idea: start with joint posterior 
density, then remove constants (e.g. terms with only fixed parameters) 
and rearrange to derive density conditional distribution we can sample 
from. To simplify notation, let $\theta[-x]$ represent all parameters 
of $\theta$ exluding parameter $x$ and let $y_i'$ represent centered data 
(e.g. $y_1' = y_1 - \mu$). \newline

Derive conditional distribution for $\sigma^2$.
\begin{align*}
    p(\sigma^2|y,\theta[-\sigma^2]) & \propto \prod_{t_i=1} N(\mu, \sigma^2 \mathbb{I}) \cdot \prod_{t_i=2} N(\gamma, \sigma^2 \mathbb{I}) \cdot \prod_{t_i=3} N(0.5\mu + 0.5\gamma, \sigma^2 \mathbb{I}) \cdot \prod_{t_i=4} N(\tau\mu + (1-\tau)\gamma, \sigma^2 \mathbb{I}) \cdot p(\sigma^2) \\
        & \propto |\sigma^2 \mathbb{I}_2|^{-n/2} \cdot \frac{1}{\sigma^2} \cdot \exp\left(-\frac{1}{2\sigma^2}\sum_{i=1}^n y_i'^T \mathbb{I}^{-1} y_i'\right) \\
        & \propto (\frac{1}{\sigma^2})^{n+1} \cdot \exp\left(-\frac{1}{\sigma^2} \cdot \frac{1}{2} \sum_{i=1}^n \|y_i'\|^2_2\right) \\
        & = x^{\alpha + 1} \cdot \exp(-x \beta) \\
    \therefore \sigma^2 | y, \theta[-\sigma^2] & \sim \text{InvGamma}\left(n, \quad \frac{1}{2} \sum_{i=1}^n \|y_i'\|^2_2\right)
\end{align*}

Derive conditional distribution for $\tau$.
\begin{align*}
    p(\tau|y,\theta[-\tau]) & \propto \exp\left(-\frac{1}{2\sigma^2} \sum_{t_i=4} (y_i - (\tau\mu + (1-\tau)\gamma))^T \mathbb{I}^{-1} (y_i - (\tau\mu + (1-\tau)\gamma)) \right)\\
        & \propto \exp\left(-\frac{1}{2} \cdot \frac{\|\gamma - \mu\|^2_2}{\sigma^2} \sum_{t_i=4} (\tau \vec{i} - \alpha_i)^T \mathbb{I}^{-1} (\tau \vec{i} - \alpha_i) \right) && \text{let $\vec{i} = (1, 1)$} \\
        & \propto \exp\left(-\frac{1}{2} \cdot \frac{\|\gamma - \mu\|^2_2}{\sigma^2} \sum_{t_i=4} \left( \|\tau \vec{i}\|^2_2 - 2\tau \vec{i}^T \alpha_i + \|\alpha_i\|^2_2\right) \right) && \text{$\|\alpha_i\|^2_2$ constant} \\
        & \propto \exp\left(-\frac{1}{2} \cdot \frac{\|\gamma - \mu\|^2_2}{\sigma^2} \left( n_4 \tau^2 - 2\tau \vec{i}^T \sum_{t_i=4} \alpha_i \right) \right) && \\
        & \propto \exp\left(-\frac{1}{2} \cdot \frac{n_4 \|\gamma - \mu\|^2_2}{\sigma^2} \left(\tau - \frac{\sum_{t_i=4} \alpha_i}{n_4} \right)^2 \right) && \text{complete the square}
\end{align*}
$$\therefore \tau | y, \theta[-\tau] \sim \text{TruncNormal}\left(\frac{\sum_{t_i=4} \alpha_i}{n_4}, \frac{\sigma^2}{n_4 \|\gamma - \mu\|^2_2}, 0, 1\right) \quad \text{where } \alpha_i = (y_i - \gamma)^T (\mu - \gamma) $$

Derive conditional distribution for $\mu$.
\begin{align*}
    p(\mu|y,\theta[-\mu]) & \propto \exp\left(-\frac{1}{2\sigma^2} \sum_{t_i \in \{1,3,4\}} (y_i - \bar{y}_i)^T \mathbb{I}^{-1} (y_i - \bar{y}_i) \right) && \text{$\bar{y}_i$ = group mean}\\
        & \propto \exp\left(-\frac{1}{2} \cdot \frac{1}{\sigma^2} \sum_{t_i \in \{1,3,4\}} (\mu - \bar{\mu}_i^{t_i})^T \mathbb{I}^{-1} (\mu - \bar{\mu}_i^{t_i}) \right) && \text{rearrange for $\mu$}\\
        & \propto \exp\left(-\frac{1}{2} \cdot \frac{1}{\sigma^2} \sum_{t_i \in \{1,3,4\}} \left(\|\mu\|^2_2 - 2\mu^T \bar{\mu}_i^{t_i} + \|\bar{\mu}_i^{t_i}\|^2_2 \right)\right) && \text{$\|\bar{\mu}_i^{t_i}\|^2_2$ constant}\\
        & \propto \exp\left(-\frac{1}{2} \cdot \frac{1}{\sigma^2} \left((n_1 + 0.5^2 n_3 + \tau^2 n_4)\|\mu\|^2_2 - 2\mu^T \sum_{t_i \in \{1,3,4\}} \bar{\mu}_i^{t_i} \right)\right) && \text{take $\|\mu\|^2_2$ outside sum}\\
        & \propto \exp\biggl(-\frac{1}{2} \cdot \frac{n_1 + 0.5^2 n_3 + \tau^2 n_4}{\sigma^2} && \text{complete the square} \\
        & \quad \left(\mu - \frac{\sum_{t_i \in \{1,3,4\}} \bar{\mu}_i^{\{t_i\}}}{n_1 + 0.5^2 n_3 + \tau^2 n_4} \right)^T \mathbb{I}^{-1} \left(\mu - \frac{\sum_{t_i \in \{1,3,4\}} \bar{\mu}_i^{\{t_i\}}}{n_1 + 0.5^2 n_3 + tau^2 n_4} \right)\biggr) && \\
    \therefore \mu | y, \theta[-\mu] & \sim \text{Normal}\left( \frac{\sum_{t_i \in \{1,3,4\}} \bar{\mu}_i^{\{t_i\}}}{n_1 + 0.5^2 n_3 + \tau^2 n_4}, \quad \frac{\sigma^2}{n_1 + 0.5^2 n_3 + \tau^2 n_4} \right) \\
    \text{where} & \quad \bar{\mu}_i^{t=1} = y_i \text{;} \quad \bar{\mu}_i^{t=3} = 0.5(y_i - 0.5\gamma) \text{;} \quad \bar{\mu}_i^{t=4} = \tau(y_i - (1-\tau)\gamma) 
\end{align*}

Derive conditional distribution for $\gamma$.
\begin{align*}
    p(\gamma|y,\theta[-\gamma]) & \propto \exp\left(-\frac{1}{2\sigma^2} \sum_{t_i \in \{2,3,4\}} (y_i - \bar{y}_i)^T \mathbb{I}^{-1} (y_i - \bar{y}_i) \right) \\ %&& \text{$\bar{y}_i$ = group mean}\\
        & \propto \exp\left(-\frac{1}{2} \cdot \frac{1}{\sigma^2} \sum_{t_i \in \{2,3,4\}} (\gamma - \bar{\gamma}_i^{t_i})^T \mathbb{I}^{-1} (\gamma - \bar{\gamma}_i^{t_i}) \right) \\ %&& \text{rearrange for $\gamma$}\\
        & \propto \exp\left(-\frac{1}{2} \cdot \frac{1}{\sigma^2} \sum_{t_i \in \{2,3,4\}} \left(\|\gamma\|^2_2 - 2\gamma^T \bar{\gamma}_i^{t_i} + \|\bar{\gamma}_i^{t_i}\|^2_2 \right)\right) \\ %&& \text{$\|\bar{\gamma}_i^{t_i}\|^2_2$ constant}\\
        & \propto \exp\left(-\frac{1}{2} \cdot \frac{1}{\sigma^2} \left((n_2 + 0.5^2 n_3 + (1-\tau)^2 n_4)\|\gamma\|^2_2 - 2\gamma^T \sum_{t_i \in \{2,3,4\}} \bar{\gamma}_i^{t_i} \right)\right) \\ %&& \text{take $\|\gamma\|^2_2$ outside sum}\\
        & \propto \exp\biggl(-\frac{1}{2} \cdot \frac{n_2 + 0.5^2 n_3 + (1-\tau)^2 n_4}{\sigma^2} \\ %&& \text{complete the square} \\
        & \quad \left(\gamma - \frac{\sum_{t_i \in \{2,3,4\}} \bar{\gamma}_i^{\{t_i\}}}{n_2 + 0.5^2 n_3 + (1-\tau)^2 n_4} \right)^T \mathbb{I}^{-1} \left(\gamma - \frac{\sum_{t_i \in \{2,3,4\}} \bar{\gamma}_i^{\{t_i\}}}{n_2 + 0.5^2 n_3 + (1-\tau)^2 n_4} \right)\biggr) && \\
    \therefore \gamma | y, \theta[-\gamma] & \sim \text{Normal}\left( \frac{\sum_{t_i \in \{2,3,4\}} \bar{\gamma}_i^{\{t_i\}}}{n_2 + 0.5^2 n_3 + (1-\tau)^2 n_4}, \quad \frac{\sigma^2}{n_2 + 0.5^2 n_3 + (1-\tau)^2 n_4} \right) \\
    \text{where} & \quad \bar{\gamma}_i^{t=2} = y_i \text{;} \quad \bar{\gamma}_i^{t=3} = 0.5(y_i - 0.5\mu) \text{;} \quad \bar{\gamma}_i^{t=4} = (1-\tau) (y_i - \tau\mu)
\end{align*}

Having derived our conditional distributions, the gibbs implementation
itself is very straightforward. We iterate through and update one or 
multiple parameters at a time (see below).  
\begin{python}
def gibbs_sampling(data, n_samples, initial_position):
    samples = [initial_position]

    it = 0
    while it < n_samples:
        it += 1
        curr_theta = samples[-1].copy()
        idx = (it-1) % 4  # systematic

        if idx == 0:
            sigma_sq_proposal = sample_ssq_conditional_posterior(data, curr_theta)
            curr_theta[0] = sigma_sq_proposal

        elif idx == 1:
            tau_proposal = sample_tau_conditional_posterior(data, curr_theta)
            curr_theta[1] = tau_proposal

        elif idx == 2:
            mu_proposal = sample_mu_conditional_posterior(data, curr_theta)
            curr_theta[2:4] = mu_proposal

        elif idx == 3:
            gam_proposal = sample_gam_conditional_posterior(data, curr_theta)
            curr_theta[4:6] = gam_proposal

        else:
            print("Error: trying to update out-of-bounds parameter!")
        samples.append(curr_theta)

    return np.array(samples)
\end{python}

The Gibbs Sampling algorithm has a number of design 
choices, including (a) systematic vs random scan, and
(b) block vs individual parameter updates. Note: there 
is no hyperparameter tuning for step size here because 
we take a completely new sample (from an adaptive proposal 
distribution) at each step.

\begin{enumerate}[label=(\alph*)]
\item \textbf{Systematic vs random scan}: While random scan ensures 
detail balance of our markov chain, I preferred to use a 
systematic scan since it was more efficient, especially over 
smaller sampling experiments (guarantees an equal number of 
updates for each).

\item \textbf{Block updates for mu and gamma}: Since $\mu_1, \mu_2$ 
and $\gamma_1, \gamma_2$ would always come from the same 
distribution I found it more efficient to update these together, 
drawing from a 2-dim multivariate normal rather than separately. 
This will not work well if the parameters within each block are 
uncorrelated but that should not be the case for this problem since 
the components of mu and gamma are closely related.
\end{enumerate}

\textbf{Results}: Using the above design choices, we produce the following 
samples from our posterior (figure 5).
\begin{figure}[H]
    \centering
    \includegraphics[scale=0.55]{gibbs_sampled_histogram.png}
    \caption{Histogram of Hamiltonian Monte Carlo posterior samples}
\end{figure}

\begin{table}[H]
    \begin{tabular}{lcccccc}
    \multicolumn{1}{c}{}          & sigma & tau   & mu1   & mu2   & gam1  & gam2  \\
    Mean of posterior samples     & 0.13  & 0.85  & -1.45 & -0.66 & -0.26 & 0.32  \\
    Variance of posterior samples & 0.001 & 0.008 & 0.016 & 0.014 & 0.022 & 0.024
    \end{tabular}
\end{table}

The traceplots in figure 6 give an indication of how well the sampler 
explored the posterior density. I am comforted by the fact that 
there are no long periods of repeated values (getting stuck) 
despite us not implementing no-turn logic. 
\begin{figure}[H]
    \centering
    \includegraphics[scale=0.55]{gibbs_sampled_traceplot.png}
    \caption{Traceplot for Hamiltonian Monte Carlo sampling}
\end{figure}

\textbf{Final remarks}. The Gibbs sampling algorithm is lightning 
fast compared to MH and HMC (25.0 sec!). This is because it does 
not execute an acceptance criterion! The main tradeoff is that our 
samples have significant autocorrelation (approx. 16 effective 
samples per 100 iterations). Note, the number of effective samples 
still surpasses MH but is almost 1/3 of our HMC result.



\section{Importance Sampling}

Importance Sampling is typically used to estimate summary statistics. 
For example, if $I_A(x)$ is the indicator function of x lying in 
some region A, and we know the target density $p(x)$, we can pick a 
proposal distribution $Q$ to sample from and weight our samples 
in the following way.
$$ E[I_A(x)] = \int I_A(x) p(x) dx = \int I_A(x) \frac{p(x)}{q(x)} q(x) dx \approx \frac{1}{n} \sum I_A(x) \frac{p(x)}{q(x)} $$ 

Adapting this theory to our problem, we want to derive our target 
density $p(\theta|y)$ (likliehood x prior) and find a suitable 
proposal distribution Q that we can sample from to produce a set 
of samples and corresponding weights for each. Scaling each sample 
by its respective weight will give us posterior samples! \newline

\begin{python}
    def multistep_importance_sampling(data, n_samples, initial_position):
    samples, weights = [initial_position], [0]
    for it in range(n_samples):

        # sample from proposal dist q
        theta = sample_from_proposal_dist(data)
        samples.append(theta)

        # evaluate p(theta) and q(theta) to compute weight
        p_theta = joint_posterior_density(data, theta)  # trial density
        q_theta = compute_proposal_density(data, theta)  # proposal
        weights.append(p_theta/q_theta)

    return np.array(samples), np.array(weights)
\end{python}

The importance sampling method requires us to make a number of design 
choices, including (a) selecting a proposal distribution, (b) whether to 
use a normalized scheme, and (c) choice of prior for $\tau$.

\begin{enumerate}[label=(\alph*)]
    \item \textbf{Proposal distribution}. Want to find a proposal distribution that is as similar 
    as possible (but with greater variance) to distribution described by our target density. 
    Idea: split evaluation of the posterior into two separate steps: 
    \begin{itemize}
        \item \textbf{Step 1}: Use only groups 1 and 2 to generate a closed form partial posterior ($\mu, \gamma, \sigma^2$ only). Combine with some $p(\tau)$ to generate our proposal distribution Q
        \item \textbf{Step 2}: Use output from step 1 as our new prior, then update using likelihood of data from groups 3 and 4 given parameters $\theta$ to generate our target density (equivalent to joint posterior) 
    \end{itemize}

    This two-step approach is attractive because it uses our data effectively to inform 
    our proposal distribution. We will also see that it simplifies our weight calculation. 
    Let $\theta = (\mu, \gamma, \tau, \sigma^2)$.
    \begin{align*}
        p(\theta|Y_{1,2,3,4}) & \propto p(\mu, \gamma, \sigma^2|Y_{1,2}) \cdot p(\tau) \cdot L(Y_{3,4}|\theta) \\
            & \propto p(\mu, \gamma, \sigma^2|Y_{1,2}) \cdot p(\tau) \cdot \prod_{t\in{3,4}} p(y_i|\theta) \\
            & \propto q(\theta) \cdot \prod_{t_i\in{3,4}} p(y_i|\theta) && \text{choose proposal $q(\theta)$}
    \end{align*}

    Want to find closed form distribution for $p(\mu, \gamma, \sigma^2|Y_{1,2})$. 
    First recognize we can separate $\sigma^2$ from $\mu$ and $\gamma$.
    $$ p(\mu, \gamma, \sigma^2|Y_{1,2}) = p(\mu, \gamma|Y_{1,2}) \cdot p(\sigma^2|\mu, \gamma, Y_{1,2}) $$
    
    Derive $p(\sigma^2|\mu, \gamma, Y_{1,2})$: $\sigma^2$ conditional on $\mu$, $\gamma$ and groups 1, 2.
    \begin{align*}
        p(\sigma^2|\mu, \gamma, Y_{1,2}) & \propto \frac{1}{\sigma^2} \cdot \left(\frac{1}{\sigma^2}\right)^{n_1 + n_2} \exp\left(-\frac{1}{2\sigma^2}(\sum_{t_i=1} \|y_i-\mu\|^2_2 + \sum_{t_i=2} \|y_i-\gamma\|^2_2)\right) \\
            & \text{let $M_{\sigma^2}=\sum_{t_i=1} \|y_i-\mu\|^2_2 + \sum_{t_i=2} \|y_i-\gamma\|^2_2$} \\
            & \propto \left(\frac{1}{\sigma^2}\right)^{n_1 + n_2 + 1} \exp\left(-\frac{M_{\sigma^2}}{2} \cdot \frac{1}{\sigma^2}\right) \\
            & \propto \text{InvGamma}\left(n_1+n_2, \quad \frac{M_{\sigma^2}}{2} \right)
    \end{align*}

    Derive $p(\mu, \gamma|Y_{1,2})$: the marginal distribution of $\mu$ and $\gamma$.
    \begin{align*}
        p(\mu, \gamma|Y_{1,2}) & = \int_0^{\infty} p(\mu, \gamma, \sigma^2|Y_{1,2}) d\sigma^2 \\
            & \propto \int_0^{\infty} \left(\frac{1}{\sigma^2}\right)^{n_1+n_2+1} \exp\left(-\frac{1}{2\sigma^2}(\sum_{t_i=1} \|y_i-\mu\|^2_2 + \sum_{t_i=2} \|y_i-\gamma\|^2_2)\right) d\sigma^2 \\
        \textrm{ recall  }\quad& \frac{\Gamma(\alpha)}{\beta^\alpha} = \int_0^\infty (\sigma^2)^{-\alpha - 1} \exp\left(\frac{\beta}{\sigma^2}\right)d\sigma^2\\
            & \propto \; \Gamma(n_1 + n_2) \left(-\frac{1}{2} ( \sum_{t_i=1} \lVert y_i - \mu\rVert^2 + \sum_{t_i=2}\lVert y_i - \gamma \rVert^2)\right)^{-(n_1 + n_2)} \\
            & \propto \; (\sum_{t_i=1}\lVert y_i - \overline{Y_1} \rVert^2 + \sum_{t_i=2}\lVert y_i - \overline{Y_2} \rVert^2 + n_1\lVert \overline{Y_1} - \mu \rVert^2 + n_2\lVert \overline{Y_2} - \gamma \rVert^2)^{-(n_1 + n_2)}\\
        \textrm{ define  }\quad& M_{\mu, \gamma} = \sum_{t_i=1}\lVert y_i - \overline{Y_1} \rVert^2 + \sum_{t_i=2}\lVert y_i - \overline{Y_2} \rVert^2\\
            & \propto \; (M_{\mu, \gamma} + n_1\lVert \overline{Y_1} - \mu \rVert^2 + n_2\lVert \overline{Y_2} - \gamma \rVert^2)^{-(n_1 + n_2)}\\
            & \propto \; \left(M_{\mu, \gamma} + \left[\left(\begin{matrix}\mu\\ \gamma\end{matrix}\right) - \left(\begin{matrix*}\overline{Y_1}\\ \overline{Y_2}\end{matrix*}\right)\right]^T \left(\begin{matrix*} n_1 & 0 & 0 & 0 \\ 0 & n_1 & 0 & 0 \\ 0 & 0 & n_2 & 0 \\ 0 & 0 & 0 & n_2 \end{matrix*}\right) \left[\left(\begin{matrix}\mu\\ \gamma\end{matrix}\right) - \left(\begin{matrix*}\overline{Y_1}\\ \overline{Y_2}\end{matrix*}\right)\right]\right)^{-(n_1 + n_2)}\\
            & \propto \; \biggl(1 + \frac{2(n_1 + n_2) - 4}{M_{\mu, \gamma}(2(n_1 + n_2) - 4)}\left[\left(\begin{matrix}\mu\\ \gamma\end{matrix}\right) - \left(\begin{matrix*}\overline{Y_1}\\ \overline{Y_2}\end{matrix*}\right)\right]^T \left(\begin{matrix*} n_1{-1} & 0 & 0 & 0 \\ 0 & n_1^{-1} & 0 & 0 \\ 0 & 0 & n_2{-1} & 0 \\ 0 & 0 & 0 & n_2{-1} \end{matrix*}\right)^{-1} \\
            & \quad \quad \quad \left[\left(\begin{matrix}\mu\\ \gamma\end{matrix}\right) - \left(\begin{matrix*}\overline{Y_1}\\ \overline{Y_2}\end{matrix*}\right)\right]\biggr)^{-(n_1 + n_2)}\\
            & \propto \; \left[1 + \frac{1}{\nu}(x - \eta)^T\Sigma^{-1}(x - \eta)\right]^{-\frac{\nu + 4}{2}} \quad \text{recognize t-distribution density}\\
        \text{ where }\quad& \nu=2(n_1 + n_2) - 4, \Sigma=\frac{M_{\mu, \gamma}}{2(n_1 + n_2) - 4}\left(\begin{matrix*} n_1{-1} & 0 & 0 & 0 \\ 0 & n_1^{-1} & 0 & 0 \\ 0 & 0 & n_2{-1} & 0 \\ 0 & 0 & 0 & n_2{-1} \end{matrix*}\right), \eta=\left(\begin{matrix*}\overline{Y_1}\\ \overline{Y_2}\end{matrix*}\right)\\
        \therefore p(\mu, \gamma \mid Y_{1,2}) &\sim t_4\left(\left(\begin{matrix*}
            \overline{Y_1}\\ \overline{Y_2} \end{matrix*}\right), \frac{M_{\mu, \gamma}}{2(n_1 + n_2) - 4}\left(\begin{matrix*}
            n_1^{-1} & 0 & 0 & 0 \\ 0 & n_1^{-1} & 0 & 0 \\ 0 & 0 & n_2^{-1} & 0 \\ 0 & 0 & 0 & n_2^{-1}
            \end{matrix*}\right)\right)
    \end{align*}

    Putting these parts together we have the following proposal distribution,
    $$ q(\theta) \sim t_4\left(\left(\begin{matrix*}
        \overline{Y_1}\\ \overline{Y_2} \end{matrix*}\right), \frac{M_{\mu, \gamma}}{2(n_1 + n_2) - 4}\left(\begin{matrix*}
        n_1^{-1} & 0 & 0 & 0 \\ 0 & n_1^{-1} & 0 & 0 \\ 0 & 0 & n_2^{-1} & 0 \\ 0 & 0 & 0 & n_2^{-1}
        \end{matrix*}\right)\right) \cdot \text{InvGamma}\left(n_1+n_2, \quad \frac{M_{\sigma^2}}{2} \right) $$

    \item \textbf{Normalized importance sampling}. Note that we cannot solve 
    for closed form distribution of our joint posterior after groups 1 and 2. 
    Idea: let our target $p(x) = f(x) / z_p$ and $q(x) = g(x) / z_q$ where 
    we can evaluate $f(x), g(x)$ but not $p(x), q(x)$ are $z_p, z_q$ are 
    constants. \newline

    If we draw $x_j$ $j=1,2,...n$ iid samples from Q distribution, then compute 
    weights $u_j = f(x_j) / g(x_j)$ and apply these to our samples, we can 
    recover the weights as if we had been able to evaluate $p(x)$ and 
    directly compute $w_j = p(x_j) / q(x_j)$:
    $$ \hat{x_j} = \frac{h(x_j) \cdot w_j}{\sum_{i=1}^n w_j} = \frac{\frac{z_p}{z_q} \cdot h(x_j) \cdot \frac{f(x_j)}{g(x_j)}}{\sum_{i=1}^n \frac{f(x_j)}{g(x_j)}} = \frac{h(x_j) \cdot u_j}{\sum_{j=1}^n u_j}$$

    This is important for our particular problem since it is not possible 
    to compute the closed form posterior (target density) and so instead 
    we need to make do with some $f(x) \propto p(x)$, which we take to be
    the likelihood x prior. \newline
    
    Note our choice of $q(\theta)$  simplies the weight calculation to just
    evaluating the likelihood of groups 3 and 4 given the samples $\theta_j$ 
    from proposal distribution $q$.
    $$ u_j = \frac{f(\theta_j)}{g(\theta_j)} = \frac{p(\theta_j|Y_{1,2,3,4})}{p(\mu_j, \gamma_j, \sigma_j^2|Y_{1,2}) \cdot p(\tau_j)} = \frac{q(\theta_j) \cdot \prod_{t_i\in{3,4}} p(y_i|\theta_j)}{q(\theta_j)} = \prod_{t_i\in{3,4}} p(y_i|\theta_j) $$

    \item \textbf{Choice of tau prior}. While we were able to solve for a 
    closed form of the posterior for groups 1 and 2, this does not give us 
    any additional information about $\tau$ to use as a prior for Step 2. 
    I tried both using a beta centered at zero $\text{Beta}(5, 5)$ and a 
    beta centered at 0.9 $\text{Beta}(20, 3)$ (closer to the MLE estimate).
    As expected the latter worked much better since it generated more 
    samples closer to our target.

\end{enumerate}

\textbf{Results}: Using the above design choices we want to generate 
the posterior marginals for our parameters $\theta$. Unlike for our other 
sampling methods, we do not get these as a direct output from the algorithm.
Idea: specify a set of binned regions ($A$) for each parameter  that cover 
the region of posterior marginal density, then use our samples $\theta_j$ 
from proposal distribution $q$ and corresponding weights $w_j$ to compute 
$\mathbb{E}[I_A(\theta)]$ for each region $A$. These "expectations of indicator 
functions" are the probabilities of a given parameter falling in region $A$, which 
we can now plot in a histogram as individual bars in our marginal posterior histogram.   
$$ \mathbb{E_{\theta \sim p}}[I_A(\theta)] = \frac{1}{n} \sum_j I_A(\theta_i) \frac{p(\theta_i)}{q(\theta_i)} = \frac{1}{n} \sum_i I_A(\theta_i) \frac{f(\theta_i)}{g(\theta_i)} = \frac{1}{n} \sum_i I_A(\theta_i) u_i$$
\begin{figure}[H]
    \centering
    \includegraphics[scale=0.55]{mis_sampled_histogram.png}
    \caption{Histogram of Importance Sampling posterior samples}
\end{figure}

\begin{table}[H]
    \begin{tabular}{lcccccc}
    \multicolumn{1}{c}{}         & sigma & tau   & mu1   & mu2   & gam1  & gam2  \\
    Mean of weighted samples     & 0.20  & 0.88  & -1.42 & -0.66 & -0.27 & 0.34  \\
    Variance of weighted samples & 0.022 & 0.004 & 0.020 & 0.077 & 0.044 & 0.058
    \end{tabular}
\end{table}

\textbf{Final remarks}. While the Importance Sampling algorithm could be highly parallelized 
(each sample is independent) we needed to run many more samples than other methods (1,000,000 
vs 5,000) to sufficiently cover our desire regions for the marginal posterior plot. Moreover, 
the algorithm was VERY senstivie to the closeness of our proposal distribution $q$. In this 
case we were able to compute a good closed form approximation from groups 1 and 2, but one 
could imagine this not generalizing well to problems with more data or more complex 
parameterizations.  

\end{document}
